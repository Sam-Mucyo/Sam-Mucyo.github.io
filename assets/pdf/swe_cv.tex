\documentclass[11pt,a4paper]{article}

\usepackage[margin=2cm]{geometry}
% \usepackage{hyperref}
\usepackage{enumitem}
\usepackage[T1]{fontenc}

\usepackage[colorlinks=true, urlcolor=blue, linkcolor=blue, pdfborder={0 0 0}]{hyperref}


% Custom commands for consistent formatting
\newcommand{\sectionheading}[1]{\vspace{0.2cm}\textbf{\Large #1}\vspace{0.1cm}\hrule\vspace{0.3cm}}
\newcommand{\subheading}[1]{\textbf{#1}}
\newcommand{\daterange}[1]{\hfill{#1}}

% Remove page numbers
\pagenumbering{gobble}

\begin{document}

\begin{center}
    \textbf{\LARGE Samuel Mucyo}\\
    \vspace{0.3cm}
    samlmucyo@gmail.com \enspace | \enspace
    \href{https://www.linkedin.com/in/mucyo-samuel}{LinkedIn}
    \enspace | \enspace
    \href{https://github.com/Sam-Mucyo}{GitHub}
    \enspace | \enspace
    \href{https://sam-mucyo.github.io/}{Personal website}
\end{center}



% \sectionheading{OBJECTIVE}
% Software engineer with experience in distributed systems and high-performance computing. Demonstrated track record of building scalable solutions at Amazon across multiple teams, with particular expertise in backend development and cloud architecture. Strong academic foundation in computer science with interest and focus in data systems and parallel computing.
\sectionheading{SUMMARY}
Senior Computer Science student with a strong background in high-performance computing (HPC), Python scripting, and collaborative software development. Experienced in automating complex workflows and building scalable systems in distributed environments. Passionate about scientific computing, data analysis, and improving end-user productivity. Eager to leverage interdisciplinary collaboration skills to help advance drug discovery and AI-driven research.

\sectionheading{EDUCATION}
\subheading{Harvard University}, Cambridge, MA \\
Bachelor of Arts in Computer Science \hfill \daterange{Expected Graduation: May 2025} \\
\textit{Relevant Courses:} High-Performance Computing (CS2050), Systems Development for Computational Science, Data Structures \& Algorithms, Systems Programming, Machine Learning, Database Systems, Data Science

\sectionheading{SKILLS}
\begin{itemize}[leftmargin=*,nosep]
    \item \textbf{Programming Languages:} Python, C++, Java, SQL
    \item \textbf{HPC \& Scientific Computing:} POSIX threads, OpenMP, MPI, GDB, Valgrind
    \item \textbf{Scripting \& Automation:} Bash, Docker, CI/CD (GitHub Actions), Linux/Unix
    \item \textbf{Data Pipelines \& Cloud:} AWS (Lambda, S3, Glue, DynamoDB), ETL workflows
    \item \textbf{Collaboration \& Communication:} Code reviews, technical documentation, Agile methodologies
\end{itemize}

\sectionheading{EXPERIENCE}

\textbf{Software Development Engineer Intern, Amazon} \hfill \daterange{May 2024 -- Aug 2024}
\begin{itemize}[leftmargin=*,nosep]
    \item Designed a Java/Spring MVC A/B testing framework to automate backend workflows, significantly reducing manual setup time.
    \item Optimized API integration, reducing unnecessary calls by 20\% and improving the statistical significance of test outcomes.
\end{itemize}

\textbf{Software Development Engineer Intern, Amazon} \hfill \daterange{May 2023 -- Aug 2023}
\begin{itemize}[leftmargin=*,nosep]
    \item Developed REST APIs in Python and AWS CDK to streamline fraud detection, cutting alert resolution times from 48 hours to 2 hours.
    \item Deployed serverless backends (Lambda, DynamoDB, EventBridge) to handle over 10K daily queries, ensuring reliable scalability.
\end{itemize}

\textbf{Software Development Engineer Intern, Amazon} \hfill \daterange{Jun 2022 -- Aug 2022}
\begin{itemize}[leftmargin=*,nosep]
    \item Architected a serverless data lake (Python, S3, Glue) to reduce test analysis time by 50\%.
    \item Integrated data pipelines with internal visualization tools, enabling fast detection of performance anomalies.
\end{itemize}

\medskip
\textbf{Teaching Fellow (part-time), Harvard University} \hfill \daterange{Sept 2022 -- May 2025}
\begin{itemize}[leftmargin=*,nosep]
    \item Conduct detailed code reviews for over 40 student projects and weekly problem sets through grading, providing feedback on code quality, efficiency, and best practices.
    \item Mentor students in weekly office hours, debugging C, C++, and Python.
    \item Led weekly sections on web development (\href{https://cs50.harvard.edu/college/2023/fall/}{CS50}) and, currently, parallel programming (\href{https://sites.google.com/g.harvard.edu/cs-2050/syllabus}{CS2050}), reinforcing fundamental concepts and practical skills.
\end{itemize}

\sectionheading{PROJECTS}

\subheading{Optimized Column-Store Database System}
\textit{Individual Project, Data Systems} \hfill \daterange{Fall 2024}
\begin{itemize}[leftmargin=*,nosep]
    \item Developed a high-performance columnar database engine in C++ with multi-threaded query execution, memory-mapped file persistence, and B-tree indexing.
    \item Improved performance on skewed data sets and achieved top-3 ranking in class benchmarks for index operations.
    \item Containerized the codebase with Docker for consistent testing and deployment.
\end{itemize}

% \subheading{Parallelizing Urban Transit Construction with Minimum Spanning Trees}
% \textit{Group Project, High-Performance Computing} \hfill \daterange{Spring 2024}
% \begin{itemize}[leftmargin=*,nosep]
%     \item Collaborated on an HPC project to implement a parallel version of Kruskal’s MST algorithm using OpenMP and MPI.
%     \item Conducted performance analysis with PAPI for hardware counter measurements, optimizing parallel execution and resource usage.
%     \item Automated strong and weak scaling studies with parameterized bash scripts, demonstrating a keen focus on reproducibility and detail.
% \end{itemize}

% \subheading{Astrolibrary}
% \textit{Group Project, Systems Development for Computational Science} \hfill \daterange{Fall 2023}
% \begin{itemize}[leftmargin=*,nosep]
%     \item Created a Python library for astronomical spectral analysis, collaborating on API contracts and software architecture.
%     \item Implemented a comprehensive test suite using \textit{pytest} and documentation with \textit{Sphinx} for maintainability.
%     \item Automated CI/CD pipelines via GitHub Actions to streamline development and deployment.
% \end{itemize}


\end{document}
